\chapter{Ansible - advanced}
\hypertarget{conditionals-handlers-and-loops}{%
\section{Conditionals, Handlers and
Loops}\label{conditionals-handlers-and-loops}}


\hypertarget{objective}{%
\subsection{Objective}\label{objective}}

Three foundational Ansible features are:

\begin{itemize}
\tightlist
\item
  \href{https://docs.ansible.com/ansible/latest/user_guide/playbooks_conditionals.html}{Conditionals}
\item
  \href{https://docs.ansible.com/ansible/latest/user_guide/playbooks_intro.html\#handlers-running-operations-on-change}{Handlers}
\item
  \href{https://docs.ansible.com/ansible/latest/user_guide/playbooks_loops.html}{Loops}
\end{itemize}

\hypertarget{guide}{%
\subsection{Guide}\label{guide}}

\hypertarget{step-1---conditionals}{%
\subsubsection{Step 1 - Conditionals}\label{step-1---conditionals}}

Ansible can use conditionals to execute tasks or plays when certain
conditions are met.

To implement a conditional, the \texttt{when} statement must be used,
followed by the condition to test. The condition is expressed using one
of the available operators like e.g.~for comparison:

\begin{longtable}[]{@{}
  >{\raggedright\arraybackslash}p{(\columnwidth - 2\tabcolsep) * \real{0.0541}}
  >{\raggedright\arraybackslash}p{(\columnwidth - 2\tabcolsep) * \real{0.9459}}@{}}
\toprule\noalign{}
\endhead
\bottomrule\noalign{}
\endlastfoot
== & Compares two objects for equality. \\
!= & Compares two objects for inequality. \\
\textgreater{} & true if the left hand side is greater than the right
hand side. \\
\textgreater= & true if the left hand side is greater or equal to the
right hand side. \\
\textless{} & true if the left hand side is lower than the right hand
side. \\
\textless= & true if the left hand side is lower or equal to the right
hand side. \\
\end{longtable}

For more on this, please refer to the documentation:
\url{http://jinja.pocoo.org/docs/2.10/templates/}

As an example you would like to install an FTP server, but only on hosts
that are in the ``ftpserver'' inventory group.

To do that, first edit the inventory to add another group, and place
\texttt{node} in it. The section to add looks like this:

\begin{Shaded}
\begin{Highlighting}[]
\KeywordTok{[ftpserver]}
\DataTypeTok{node}
\end{Highlighting}
\end{Shaded}

Edit the inventory \texttt{\textasciitilde{}/hosts} to
add those lines. When you are done, it will look similar to the
following listing:

\begin{quote}
\textbf{Tip}

The ansible\_host variable only needs to be specified once for a node.
When adding a node to other groups, you do not need to specify the
variable again.
\end{quote}

\textbf{Important} Do not copy/paste the example below. Just edit the
file to add the above lines.

\begin{Shaded}
\begin{Highlighting}[]
\KeywordTok{[web]}
\DataTypeTok{node ansible\_host}\OtherTok{=}\StringTok{xx.xx.xx.xx} \DataTypeTok{ansible\_user}\OtherTok{=}\StringTok{[USER]}

\KeywordTok{[ftpserver]}
\DataTypeTok{node}

\KeywordTok{[control]}
\DataTypeTok{controller ansible\_host}\OtherTok{=}\StringTok{xx.xx.xx.xx} \DataTypeTok{ansible\_user}\OtherTok{=}\StringTok{[USER]}
\end{Highlighting}
\end{Shaded}

Next create the file \texttt{ftpserver.yml} on your control host in the
\texttt{\textasciitilde{}/ansible-files/} directory:

\begin{Shaded}
\begin{Highlighting}[]
\PreprocessorTok{{-}{-}{-}}
\KeywordTok{{-}}\AttributeTok{ }\FunctionTok{name}\KeywordTok{:}\AttributeTok{ Install vsftpd on ftpservers}
\AttributeTok{  }\FunctionTok{hosts}\KeywordTok{:}\AttributeTok{ all}
\AttributeTok{  }\FunctionTok{become}\KeywordTok{:}\AttributeTok{ }\CharTok{True}
\AttributeTok{  }\FunctionTok{tasks}\KeywordTok{:}
\AttributeTok{    }\KeywordTok{{-}}\AttributeTok{ }\FunctionTok{name}\KeywordTok{:}\AttributeTok{ Install FTP server when host in ftpserver group}
\AttributeTok{      }\FunctionTok{ansible.builtin.yum}\KeywordTok{:}
\AttributeTok{        }\FunctionTok{name}\KeywordTok{:}\AttributeTok{ vsftpd}
\AttributeTok{        }\FunctionTok{state}\KeywordTok{:}\AttributeTok{ latest}
\AttributeTok{      }\FunctionTok{when}\KeywordTok{:}\AttributeTok{ inventory\_hostname in groups["ftpserver"]}
\end{Highlighting}
\end{Shaded}

\begin{quote}
\textbf{Tip}

By now you should know how to run Ansible Playbooks, we'll start to be
less verbose in this guide. Go create and run it. :-)
\end{quote}

Run it and examine the output. The expected outcome: The task is skipped
on the ansible host (your control host) because they
are not in the ftpserver group in your inventory file.

\begin{Shaded}
\begin{Highlighting}[]
\ExtensionTok{TASK}\NormalTok{ [Install FTP server when host in ftpserver group] }\PreprocessorTok{**************}
\ExtensionTok{skipping:} \PreprocessorTok{[}\SpecialStringTok{controller}\PreprocessorTok{]}
\ExtensionTok{changed:} \PreprocessorTok{[}\SpecialStringTok{node}\PreprocessorTok{]}
\end{Highlighting}
\end{Shaded}

\hypertarget{step-2---handlers}{%
\subsubsection{Step 2 - Handlers}\label{step-2---handlers}}

Sometimes when a task does make a change to the system, an additional
task or tasks may need to be run. For example, a change to a service's
configuration file may then require that the service be restarted so
that the changed configuration takes effect.

Here Ansible's handlers come into play. Handlers can be seen as inactive
tasks that only get triggered when explicitly invoked using the
``notify'' statement. Read more about them in the
\href{http://docs.ansible.com/ansible/latest/playbooks_intro.html\#handlers-running-operations-on-change}{Ansible
Handlers} documentation.

As a an example, let's write a playbook that:

\begin{itemize}
\tightlist
\item
  manages Apache's configuration file
  \texttt{/etc/httpd/conf/httpd.conf} on all hosts in the \texttt{web}
  group
\item
  restarts Apache when the file has changed
\end{itemize}

First we need the file Ansible will deploy, let's just take the one from
node1. Remember to replace the IP address shown in the listing below
with the IP address from your individual \texttt{node1}.

\begin{Shaded}
\begin{Highlighting}[]
\ExtensionTok{[student@controller}\NormalTok{ ansible{-}files]$ scp 10.3.48.[100+PARTICIPANT\_ID]:/etc/httpd/conf/httpd.conf \\}
\ExtensionTok{\textasciitilde{}/ansible{-}files/files/httpd.conf}
\end{Highlighting}
\end{Shaded}

Next, create the Playbook \texttt{httpd\_conf.yml}. Make sure that you
are in the directory \texttt{\textasciitilde{}/ansible-files}.

\begin{Shaded}
\begin{Highlighting}[]
\PreprocessorTok{{-}{-}{-}}
\KeywordTok{{-}}\AttributeTok{ }\FunctionTok{name}\KeywordTok{:}\AttributeTok{ Manage httpd.conf}
\AttributeTok{  }\FunctionTok{hosts}\KeywordTok{:}\AttributeTok{ web}
\AttributeTok{  }\FunctionTok{become}\KeywordTok{:}\AttributeTok{ }\CharTok{True}
\AttributeTok{  }\FunctionTok{tasks}\KeywordTok{:}
\AttributeTok{    }\KeywordTok{{-}}\AttributeTok{ }\FunctionTok{name}\KeywordTok{:}\AttributeTok{ Copy Apache configuration file}
\AttributeTok{      }\FunctionTok{ansible.builtin.copy}\KeywordTok{:}
\AttributeTok{        }\FunctionTok{src}\KeywordTok{:}\AttributeTok{ httpd.conf}
\AttributeTok{        }\FunctionTok{dest}\KeywordTok{:}\AttributeTok{ /etc/httpd/conf/}
\AttributeTok{        }\FunctionTok{mode}\KeywordTok{:}\AttributeTok{ }\StringTok{\textquotesingle{}644\textquotesingle{}}
\AttributeTok{      }\FunctionTok{notify}\KeywordTok{:}
\AttributeTok{        }\KeywordTok{{-}}\AttributeTok{ Restart\_apache}
\AttributeTok{    }\KeywordTok{{-}}\AttributeTok{ }\FunctionTok{name}\KeywordTok{:}\AttributeTok{ Open firewall port}
\AttributeTok{      }\FunctionTok{ansible.posix.firewalld}\KeywordTok{:}
\AttributeTok{        }\FunctionTok{port}\KeywordTok{:}\AttributeTok{ 8080/tcp}
\AttributeTok{        }\FunctionTok{immediate}\KeywordTok{:}\CharTok{ True}
\AttributeTok{        }\FunctionTok{permanent}\KeywordTok{:}\CharTok{ True}
\AttributeTok{        }\FunctionTok{state}\KeywordTok{:}\AttributeTok{ enabled}
\AttributeTok{  }\FunctionTok{handlers}\KeywordTok{:}
\AttributeTok{    }\KeywordTok{{-}}\AttributeTok{ }\FunctionTok{name}\KeywordTok{:}\AttributeTok{ Restart\_apache}
\AttributeTok{      }\FunctionTok{ansible.builtin.service}\KeywordTok{:}
\AttributeTok{        }\FunctionTok{name}\KeywordTok{:}\AttributeTok{ httpd}
\AttributeTok{        }\FunctionTok{state}\KeywordTok{:}\AttributeTok{ restarted}
\end{Highlighting}
\end{Shaded}

So what's new here?

\begin{itemize}
\tightlist
\item
  The ``notify'' section calls the handler only when the copy task
  actually changes the file. That way the service is only restarted if
  needed - and not each time the playbook is run.
\item
  The ``handlers'' section defines a task that is only run on
  notification.
\end{itemize}

Run the playbook. We didn't change anything in the file yet so there
should not be any \texttt{changed} lines in the output and of course the
handler shouldn't have fired.

\begin{itemize}
\tightlist
\item
  Now change the \texttt{Listen\ 80} line in
  \texttt{\textasciitilde{}/ansible-files/files/httpd.conf} to:
\end{itemize}

\begin{Shaded}
\begin{Highlighting}[]
\DataTypeTok{Listen 8080}
\end{Highlighting}
\end{Shaded}

\begin{itemize}
\item
  Run the playbook again. Now the Ansible's output should be a lot more
  interesting:

  \begin{itemize}
  \tightlist
  \item
    httpd.conf should have been copied over
  \item
    The handler should have restarted Apache
  \end{itemize}
\end{itemize}

Apache should now listen on port 8080. Easy enough to verify:

\begin{Shaded}
\begin{Highlighting}[]
\ExtensionTok{[student@controller}\NormalTok{ ansible{-}files]$ curl http://10.3.48.[100+PARTICIPANT\_ID]}
\ExtensionTok{curl:} \ExtensionTok{(7)} \ExtensionTok{Failed}\NormalTok{ to connect to 10.3.48.101 port 80: Connection refused}
\ExtensionTok{[student@controller}\NormalTok{ ansible{-}files]$ curl http://10.3.48.[100+PARTICIPANT\_ID]:8080}
\OperatorTok{\textless{}}\NormalTok{body}\OperatorTok{\textgreater{}}
\OperatorTok{  \textless{}}\NormalTok{h1}\OperatorTok{\textgreater{}}\NormalTok{Apache }\ExtensionTok{is}\NormalTok{ running fine}\OperatorTok{\textless{}}\NormalTok{/h1}\OperatorTok{\textgreater{}}
\OperatorTok{\textless{}}\NormalTok{/body}\OperatorTok{\textgreater{}}
\end{Highlighting}
\end{Shaded}

Leave the setting for listen on port 8080. We'll use this in a later
exercise.

\hypertarget{step-3---simple-loops}{%
\subsubsection{Step 3 - Simple Loops}\label{step-3---simple-loops}}

Loops enable us to repeat the same task over and over again. For
example, lets say you want to create multiple users. By using an Ansible
loop, you can do that in a single task. Loops can also iterate over more
than just basic lists. For example, if you have a list of users with
their coresponding group, loop can iterate over them as well. Find out
more about loops in the
\href{https://docs.ansible.com/ansible/latest/user_guide/playbooks_loops.html}{Ansible
Loops} documentation.

To show the loops feature we will generate three new users on
\texttt{node}. For that, create the file \texttt{loop\_users.yml} in
\texttt{\textasciitilde{}/ansible-files} on your control node as your
student user. We will use the \texttt{user} module to generate the user
accounts.

\begin{Shaded}
\begin{Highlighting}[]
\PreprocessorTok{{-}{-}{-}}
\KeywordTok{{-}}\AttributeTok{ }\FunctionTok{name}\KeywordTok{:}\AttributeTok{ Ensure users}
\AttributeTok{  }\FunctionTok{hosts}\KeywordTok{:}\AttributeTok{ node}
\AttributeTok{  }\FunctionTok{become}\KeywordTok{:}\AttributeTok{ }\CharTok{True}
\AttributeTok{  }\FunctionTok{tasks}\KeywordTok{:}
\AttributeTok{    }\KeywordTok{{-}}\AttributeTok{ }\FunctionTok{name}\KeywordTok{:}\AttributeTok{ Ensure three users are present}
\AttributeTok{      }\FunctionTok{ansible.builtin.user}\KeywordTok{:}
\AttributeTok{        }\FunctionTok{name}\KeywordTok{:}\AttributeTok{ }\StringTok{"\{\{ item \}\}"}
\AttributeTok{        }\FunctionTok{state}\KeywordTok{:}\AttributeTok{ present}
\AttributeTok{      }\FunctionTok{loop}\KeywordTok{:}
\AttributeTok{         }\KeywordTok{{-}}\AttributeTok{ dev\_user}
\AttributeTok{         }\KeywordTok{{-}}\AttributeTok{ qa\_user}
\AttributeTok{         }\KeywordTok{{-}}\AttributeTok{ prod\_user}
\end{Highlighting}
\end{Shaded}

Understand the playbook and the output:

\begin{itemize}
\tightlist
\item
  The names are not provided to the user module directly. Instead, there
  is only a variable called \texttt{\{\{\ item\ \}\}} for the parameter
  \texttt{name}.
\item
  The \texttt{loop} keyword lists the actual user names. Those replace
  the \texttt{\{\{\ item\ \}\}} during the actual execution of the
  playbook.
\item
  During execution the task is only listed once, but there are three
  changes listed underneath it.
\end{itemize}

\hypertarget{step-4---loops-over-hashes}{%
\subsubsection{Step 4 - Loops over
hashes}\label{step-4---loops-over-hashes}}

As mentioned loops can also be over lists of hashes. Imagine that the
users should be assigned to different additional groups:

\begin{Shaded}
\begin{Highlighting}[]
\KeywordTok{{-}}\AttributeTok{ }\FunctionTok{username}\KeywordTok{:}\AttributeTok{ dev\_user}
\AttributeTok{  }\FunctionTok{groups}\KeywordTok{:}\AttributeTok{ ftp}
\KeywordTok{{-}}\AttributeTok{ }\FunctionTok{username}\KeywordTok{:}\AttributeTok{ qa\_user}
\AttributeTok{  }\FunctionTok{groups}\KeywordTok{:}\AttributeTok{ ftp}
\KeywordTok{{-}}\AttributeTok{ }\FunctionTok{username}\KeywordTok{:}\AttributeTok{ prod\_user}
\AttributeTok{  }\FunctionTok{groups}\KeywordTok{:}\AttributeTok{ apache}
\end{Highlighting}
\end{Shaded}

The \texttt{user} module has the optional parameter \texttt{groups} to
list additional users. To reference items in a hash, the
\texttt{\{\{\ item\ \}\}} keyword needs to reference the subkey:
\texttt{\{\{\ item.groups\ \}\}} for example.

Let's rewrite the playbook to create the users with additional user
rights:

\begin{Shaded}
\begin{Highlighting}[]
\PreprocessorTok{{-}{-}{-}}
\KeywordTok{{-}}\AttributeTok{ }\FunctionTok{name}\KeywordTok{:}\AttributeTok{ Ensure users}
\AttributeTok{  }\FunctionTok{hosts}\KeywordTok{:}\AttributeTok{ node}
\AttributeTok{  }\FunctionTok{become}\KeywordTok{:}\AttributeTok{ }\CharTok{True}
\AttributeTok{  }\FunctionTok{tasks}\KeywordTok{:}
\AttributeTok{    }\KeywordTok{{-}}\AttributeTok{ }\FunctionTok{name}\KeywordTok{:}\AttributeTok{ Ensure three users are present}
\AttributeTok{      }\FunctionTok{ansible.builtin.user}\KeywordTok{:}
\AttributeTok{        }\FunctionTok{name}\KeywordTok{:}\AttributeTok{ }\StringTok{"\{\{ item.username \}\}"}
\AttributeTok{        }\FunctionTok{state}\KeywordTok{:}\AttributeTok{ present}
\AttributeTok{        }\FunctionTok{groups}\KeywordTok{:}\AttributeTok{ }\StringTok{"\{\{ item.groups \}\}"}
\AttributeTok{      }\FunctionTok{loop}\KeywordTok{:}
\AttributeTok{        }\KeywordTok{{-}}\AttributeTok{ }\KeywordTok{\{}\AttributeTok{ }\FunctionTok{username}\KeywordTok{:}\AttributeTok{ }\StringTok{\textquotesingle{}dev\_user\textquotesingle{}}\KeywordTok{,}\AttributeTok{ }\FunctionTok{groups}\KeywordTok{:}\AttributeTok{ }\StringTok{\textquotesingle{}ftp\textquotesingle{}}\AttributeTok{ }\KeywordTok{\}}
\AttributeTok{        }\KeywordTok{{-}}\AttributeTok{ }\KeywordTok{\{}\AttributeTok{ }\FunctionTok{username}\KeywordTok{:}\AttributeTok{ }\StringTok{\textquotesingle{}qa\_user\textquotesingle{}}\KeywordTok{,}\AttributeTok{ }\FunctionTok{groups}\KeywordTok{:}\AttributeTok{ }\StringTok{\textquotesingle{}ftp\textquotesingle{}}\AttributeTok{ }\KeywordTok{\}}
\AttributeTok{        }\KeywordTok{{-}}\AttributeTok{ }\KeywordTok{\{}\AttributeTok{ }\FunctionTok{username}\KeywordTok{:}\AttributeTok{ }\StringTok{\textquotesingle{}prod\_user\textquotesingle{}}\KeywordTok{,}\AttributeTok{ }\FunctionTok{groups}\KeywordTok{:}\AttributeTok{ }\StringTok{\textquotesingle{}apache\textquotesingle{}}\AttributeTok{ }\KeywordTok{\}}
\end{Highlighting}
\end{Shaded}

Check the output:

\begin{itemize}
\tightlist
\item
  Again the task is listed once, but three changes are listed. Each loop
  with its content is shown.
\end{itemize}

Verify that the user \texttt{dev\_user} was indeed created on
\texttt{node} using the following playbook:

\begin{Shaded}
\begin{Highlighting}[]
\PreprocessorTok{{-}{-}{-}}
\KeywordTok{{-}}\AttributeTok{ }\FunctionTok{name}\KeywordTok{:}\AttributeTok{ Get user ID}
\AttributeTok{  }\FunctionTok{hosts}\KeywordTok{:}\AttributeTok{ node}
\AttributeTok{  }\FunctionTok{vars}\KeywordTok{:}
\AttributeTok{    }\FunctionTok{myuser}\KeywordTok{:}\AttributeTok{ }\StringTok{"dev\_user"}
\AttributeTok{  }\FunctionTok{tasks}\KeywordTok{:}
\AttributeTok{    }\KeywordTok{{-}}\AttributeTok{ }\FunctionTok{name}\KeywordTok{:}\AttributeTok{ Get \{\{ myuser \}\} info}
\AttributeTok{      }\FunctionTok{ansible.builtin.getent}\KeywordTok{:}
\AttributeTok{        }\FunctionTok{database}\KeywordTok{:}\AttributeTok{ passwd}
\AttributeTok{        }\FunctionTok{key}\KeywordTok{:}\AttributeTok{ }\StringTok{"\{\{ myuser \}\}"}
\AttributeTok{    }\KeywordTok{{-}}\AttributeTok{ }\FunctionTok{ansible.builtin.debug}\KeywordTok{:}
\AttributeTok{        }\FunctionTok{msg}\KeywordTok{:}
\AttributeTok{          }\KeywordTok{{-}}\AttributeTok{ }\StringTok{"\{\{ myuser \}\} uid: \{\{ getent\_passwd[myuser].1 \}\}"}
\end{Highlighting}
\end{Shaded}

\begin{Shaded}
\begin{Highlighting}[]
\ExtensionTok{$}\NormalTok{ ansible{-}navigator run user\_id.yml }\AttributeTok{{-}m}\NormalTok{ stdout}

\ExtensionTok{PLAY}\NormalTok{ [Get user ID] }\PreprocessorTok{*************************************************************}

\ExtensionTok{TASK}\NormalTok{ [Gathering Facts] }\PreprocessorTok{*********************************************************}
\ExtensionTok{ok:} \PreprocessorTok{[}\SpecialStringTok{node}\PreprocessorTok{]}

\ExtensionTok{TASK}\NormalTok{ [Get dev\_user info] }\PreprocessorTok{*******************************************************}
\ExtensionTok{ok:} \PreprocessorTok{[}\SpecialStringTok{node}\PreprocessorTok{]}

\ExtensionTok{TASK} \PreprocessorTok{[}\SpecialStringTok{debug}\PreprocessorTok{]} \PreprocessorTok{*******************************************************************}
\ExtensionTok{ok:} \PreprocessorTok{[}\SpecialStringTok{node}\PreprocessorTok{]}\NormalTok{ =}\OperatorTok{\textgreater{}}\NormalTok{ \{}
    \StringTok{"msg"}\ExtensionTok{:}\NormalTok{ [}
        \StringTok{"dev\_user uid: 1002"}
    \ExtensionTok{]}
\ErrorTok{\}}

\ExtensionTok{PLAY}\NormalTok{ RECAP }\PreprocessorTok{*********************************************************************}
\ExtensionTok{node}\NormalTok{: ok=3  changed=0  unreachable=0  failed=0  skipped=0  rescued=0  ignored=0}
\end{Highlighting}
\end{Shaded}

\hypertarget{templates}{%
\section{Templates}\label{templates}}

\hypertarget{objective}{%
\subsection{Objective}\label{objective}}

This exercise will cover Jinja2 templating. Ansible uses Jinja2
templating to modify files before they are distributed to managed hosts.
Jinja2 is one of the most used template engines for Python
(\url{http://jinja.pocoo.org/}).

\hypertarget{guide}{%
\subsection{Guide}\label{guide}}

\hypertarget{step-1---using-templates-in-playbooks}{%
\subsubsection{Step 1 - Using Templates in
Playbooks}\label{step-1---using-templates-in-playbooks}}

When a template for a file has been created, it can be deployed to the
managed hosts using the \texttt{template} module, which supports the
transfer of a local file from the control node to the managed hosts.

As an example of using templates you will change the motd file to
contain host-specific data.

First create the directory \texttt{templates} to hold template resources
in \texttt{\textasciitilde{}/ansible-files/}:

\begin{Shaded}
\begin{Highlighting}[]
\ExtensionTok{[student@controller}\NormalTok{ ansible{-}files]$ mkdir templates}
\end{Highlighting}
\end{Shaded}

Then in the \texttt{\textasciitilde{}/ansible-files/templates/}
directory create the template file \texttt{motd-facts.j2}:

\begin{Shaded}
\begin{Highlighting}[]
\NormalTok{Welcome to \{\{ ansible\_hostname \}\}.}
\NormalTok{\{\{ ansible\_distribution \}\} \{\{ ansible\_distribution\_version\}\}}
\NormalTok{deployed on \{\{ ansible\_architecture \}\} architecture.}
\end{Highlighting}
\end{Shaded}

The template file contains the basic text that will later be copied
over. It also contains variables which will be replaced on the target
machines individually.

Next we need a playbook to use this template. In the
\texttt{\textasciitilde{}/ansible-files/} directory create the Playbook
\texttt{motd-facts.yml}:

\begin{Shaded}
\begin{Highlighting}[]
\PreprocessorTok{{-}{-}{-}}
\KeywordTok{{-}}\AttributeTok{ }\FunctionTok{name}\KeywordTok{:}\AttributeTok{ Fill motd file with host data}
\AttributeTok{  }\FunctionTok{hosts}\KeywordTok{:}\AttributeTok{ node}
\AttributeTok{  }\FunctionTok{become}\KeywordTok{:}\AttributeTok{ }\CharTok{True}
\AttributeTok{  }\FunctionTok{tasks}\KeywordTok{:}
\AttributeTok{    }\KeywordTok{{-}}\AttributeTok{ }\FunctionTok{ansible.builtin.template}\KeywordTok{:}
\AttributeTok{        }\FunctionTok{src}\KeywordTok{:}\AttributeTok{ motd{-}facts.j2}
\AttributeTok{        }\FunctionTok{dest}\KeywordTok{:}\AttributeTok{ /etc/motd}
\AttributeTok{        }\FunctionTok{owner}\KeywordTok{:}\AttributeTok{ root}
\AttributeTok{        }\FunctionTok{group}\KeywordTok{:}\AttributeTok{ root}
\AttributeTok{        }\FunctionTok{mode}\KeywordTok{:}\AttributeTok{ }\DecValTok{0644}
\end{Highlighting}
\end{Shaded}

You have done this a couple of times by now:

\begin{itemize}
\tightlist
\item
  Understand what the Playbook does.
\item
  Execute the Playbook \texttt{motd-facts.yml}.
\item
  Login to node via SSH and check the message of the day content.
\item
  Log out of node.
\end{itemize}

You should see how Ansible replaces the variables with the facts it
discovered from the system.

\hypertarget{step-2---challenge-lab}{%
\subsubsection{Step 2 - Challenge Lab}\label{step-2---challenge-lab}}

Add a line to the template to list the current kernel of the managed
node.

\begin{itemize}
\tightlist
\item
  Find a fact that contains the kernel version using the commands you
  learned in the ``Ansible Facts'' chapter.
\end{itemize}

\begin{quote}
\textbf{Tip}

filter for kernel
\end{quote}

\begin{quote}
Run the newly created playbook to find the fact name.
\end{quote}

\begin{itemize}
\item
  Change the template to use the fact you found.
\item
  Run the motd playbook again.
\item
  Check motd by logging in to node
\end{itemize}

\begin{quote}
\textbf{Warning}

\textbf{Solution below!}
\end{quote}

\begin{itemize}
\tightlist
\item
  Find the fact:
\end{itemize}

\begin{Shaded}
\begin{Highlighting}[]
\PreprocessorTok{{-}{-}{-}}
\KeywordTok{{-}}\AttributeTok{ }\FunctionTok{name}\KeywordTok{:}\AttributeTok{ Capture Kernel Version}
\AttributeTok{  }\FunctionTok{hosts}\KeywordTok{:}\AttributeTok{ node}
\AttributeTok{  }\FunctionTok{tasks}\KeywordTok{:}
\AttributeTok{    }\KeywordTok{{-}}\AttributeTok{ }\FunctionTok{name}\KeywordTok{:}\AttributeTok{ Collect only kernel facts}
\AttributeTok{      }\FunctionTok{ansible.builtin.setup}\KeywordTok{:}
\AttributeTok{        }\FunctionTok{filter}\KeywordTok{:}
\AttributeTok{          }\KeywordTok{{-}}\AttributeTok{ }\StringTok{\textquotesingle{}*kernel\textquotesingle{}}
\AttributeTok{      }\FunctionTok{register}\KeywordTok{:}\AttributeTok{ setup}
\AttributeTok{    }\KeywordTok{{-}}\AttributeTok{ }\FunctionTok{ansible.builtin.debug}\KeywordTok{:}
\AttributeTok{        }\FunctionTok{var}\KeywordTok{:}\AttributeTok{ setup}
\end{Highlighting}
\end{Shaded}

With the wildcard in place, the output shows:

\begin{Shaded}
\begin{Highlighting}[]

\ExtensionTok{TASK} \PreprocessorTok{[}\SpecialStringTok{debug}\PreprocessorTok{]} \PreprocessorTok{*******************************************************************}
\ExtensionTok{ok:} \PreprocessorTok{[}\SpecialStringTok{node1}\PreprocessorTok{]}\NormalTok{ =}\OperatorTok{\textgreater{}}\NormalTok{ \{}
    \StringTok{"setup"}\ExtensionTok{:}\NormalTok{ \{}
        \StringTok{"ansible\_facts"}\ExtensionTok{:}\NormalTok{ \{}
            \StringTok{"ansible\_kernel"}\ExtensionTok{:} \StringTok{"4.18.0{-}513.11.1.el8\_9.ppc64le"}
        \ErrorTok{\}}\ExtensionTok{,}
        \StringTok{"changed"}\ExtensionTok{:}\NormalTok{ false,}
        \StringTok{"failed"}\ExtensionTok{:}\NormalTok{ false}
    \ErrorTok{\}}
\ErrorTok{\}}
\end{Highlighting}
\end{Shaded}

With this we can conclude the variable we are looking for is labeled
\texttt{ansible\_kernel}.

Then we can update the motd-facts.j2 template to include
\texttt{ansible\_kernel} as part of its message.

\begin{itemize}
\tightlist
\item
  Modify the template \texttt{motd-facts.j2}:
\end{itemize}

\begin{Shaded}
\begin{Highlighting}[]
\NormalTok{Welcome to \{\{ ansible\_hostname \}\}.}
\NormalTok{\{\{ ansible\_distribution \}\} \{\{ ansible\_distribution\_version\}\}}
\NormalTok{deployed on \{\{ ansible\_architecture \}\} architecture}
\NormalTok{running kernel \{\{ ansible\_kernel \}\}.}
\end{Highlighting}
\end{Shaded}

\begin{itemize}
\tightlist
\item
  Run the playbook.
\end{itemize}

\begin{Shaded}
\begin{Highlighting}[]
\ExtensionTok{[student@controller}\NormalTok{ \textasciitilde{}]$ ansible{-}navigator run motd{-}facts.yml }\AttributeTok{{-}m}\NormalTok{ stdout}
\end{Highlighting}
\end{Shaded}

\begin{itemize}
\tightlist
\item
  Verify the new message via SSH login to \texttt{node}.
\end{itemize}

\begin{Shaded}
\begin{Highlighting}[]
\ExtensionTok{[student@controller}\NormalTok{ \textasciitilde{}]$ ssh 10.3.48.[100+PARTICIPANT\_ID]}
\ExtensionTok{Welcome}\NormalTok{ to node.}
\ExtensionTok{RedHat}\NormalTok{ 8.9}
\ExtensionTok{deployed}\NormalTok{ on ppc64le architecture}
\ExtensionTok{running}\NormalTok{ kernel 4.18.0{-}513.11.1.el8\_9.ppc64le.}
\end{Highlighting}
\end{Shaded}

\hypertarget{roles---making-your-playbooks-reusable}{%
\section{Roles - Making your playbooks
reusable}\label{roles---making-your-playbooks-reusable}}

\hypertarget{objective}{%
\subsection{Objective}\label{objective}}

While it is possible to write a playbook in one file as we've done
throughout this workshop, eventually you'll want to reuse files and
start to organize things.

Ansible Roles are the way we do this. When you create a role, you
deconstruct your playbook into parts and those parts sit in a directory
structure. This is explained in more details in the
\href{https://docs.ansible.com/ansible/latest/user_guide/playbooks_best_practices.html}{Tips
and tricks} and the
\href{https://docs.ansible.com/ansible/latest/user_guide/sample_setup.html}{Sample
Ansible setup}.

This exercise will cover:

\begin{itemize}
\tightlist
\item
  the folder structure of an Ansible Role
\item
  how to build an Ansible Role
\item
  creating an Ansible Play to use and execute a role
\item
  using Ansible to create a Apache VirtualHost on node
\end{itemize}

\hypertarget{guide}{%
\subsection{Guide}\label{guide}}

\hypertarget{step-1---understanding-the-ansible-role-structure}{%
\subsubsection{Step 1 - Understanding the Ansible Role
Structure}\label{step-1---understanding-the-ansible-role-structure}}

Roles follow a defined directory structure; a role is named by the top
level directory. Some of the subdirectories contain YAML files, named
\texttt{main.yml}. The files and templates subdirectories can contain
objects referenced by the YAML files.

An example project structure could look like this, the name of the role
would be ``apache'':

\begin{Shaded}
\begin{Highlighting}[]
\NormalTok{apache/}
\NormalTok{├── defaults}
\NormalTok{│   └── main.yml}
\NormalTok{├── files}
\NormalTok{├── handlers}
\NormalTok{│   └── main.yml}
\NormalTok{├── meta}
\NormalTok{│   └── main.yml}
\NormalTok{├── README.md}
\NormalTok{├── tasks}
\NormalTok{│   └── main.yml}
\NormalTok{├── templates}
\NormalTok{├── tests}
\NormalTok{│   ├── inventory}
\NormalTok{│   └── test.yml}
\NormalTok{└── vars}
\NormalTok{    └── main.yml}
\end{Highlighting}
\end{Shaded}

The various \texttt{main.yml} files contain content depending on their
location in the directory structure shown above. For instance,
\texttt{vars/main.yml} references variables, \texttt{handlers/main.yaml}
describes handlers, and so on. Note that in contrast to playbooks, the
\texttt{main.yml} files only contain the specific content and not
additional playbook information like hosts, \texttt{become} or other
keywords.

\begin{quote}
\textbf{Tip}

There are actually two directories for variables: \texttt{vars} and
\texttt{default}. Default variables, \texttt{defaults/main.yml}, have
the lowest precedence and usually contain default values set by the role
authors and are often used when it is intended that their values will be
overridden. Variables set in \texttt{vars/main.yml} are for variables
not intended to be modified.
\end{quote}

Using roles in a Playbook is straight forward:

\begin{Shaded}
\begin{Highlighting}[]
\PreprocessorTok{{-}{-}{-}}
\KeywordTok{{-}}\AttributeTok{ }\FunctionTok{name}\KeywordTok{:}\AttributeTok{ launch roles}
\AttributeTok{  }\FunctionTok{hosts}\KeywordTok{:}\AttributeTok{ web}
\AttributeTok{  }\FunctionTok{roles}\KeywordTok{:}
\AttributeTok{    }\KeywordTok{{-}}\AttributeTok{ role1}
\AttributeTok{    }\KeywordTok{{-}}\AttributeTok{ role2}
\end{Highlighting}
\end{Shaded}

For each role, the tasks, handlers and variables of that role will be
included in the Playbook, in that order. Any copy, script, template, or
include tasks in the role can reference the relevant files, templates,
or tasks \emph{without absolute or relative path names}. Ansible will
look for them in the role's files, templates, or tasks respectively,
based on their use.

\hypertarget{step-2---create-a-basic-role-directory-structure}{%
\subsubsection{Step 2 - Create a Basic Role Directory
Structure}\label{step-2---create-a-basic-role-directory-structure}}

Ansible looks for roles in a subdirectory called \texttt{roles} in the
project directory. This can be overridden in the Ansible configuration.
Each role has its own directory. To ease creation of a new role the tool
\texttt{ansible-galaxy} can be used.

\begin{quote}
\textbf{Tip}

Ansible Galaxy is your hub for finding, reusing and sharing the best
Ansible content. \texttt{ansible-galaxy} helps to interact with Ansible
Galaxy. For now we'll just using it as a helper to build the directory
structure.
\end{quote}

Okay, lets start to build a role. We'll build a role that installs and
configures Apache to serve a virtual host. Run these commands in your
\texttt{\textasciitilde{}/ansible-files} directory:

\begin{Shaded}
\begin{Highlighting}[]
\ExtensionTok{[student@controller}\NormalTok{ ansible{-}files]$ mkdir roles}
\ExtensionTok{[student@controller}\NormalTok{ ansible{-}files]$ ansible{-}galaxy init }\AttributeTok{{-}{-}offline}\NormalTok{ roles/apache\_vhost}
\end{Highlighting}
\end{Shaded}

Have a look at the role directories and their content:

\begin{Shaded}
\begin{Highlighting}[]
\ExtensionTok{[student@controller}\NormalTok{ ansible{-}files]$ tree roles}
\end{Highlighting}
\end{Shaded}

\begin{Shaded}
\begin{Highlighting}[]
\NormalTok{roles/}
\NormalTok{└── apache\_vhost}
\NormalTok{    ├── defaults}
\NormalTok{    │   └── main.yml}
\NormalTok{    ├── files}
\NormalTok{    ├── handlers}
\NormalTok{    │   └── main.yml}
\NormalTok{    ├── meta}
\NormalTok{    │   └── main.yml}
\NormalTok{    ├── README.md}
\NormalTok{    ├── tasks}
\NormalTok{    │   └── main.yml}
\NormalTok{    ├── templates}
\NormalTok{    ├── tests}
\NormalTok{    │   ├── inventory}
\NormalTok{    │   └── test.yml}
\NormalTok{    └── vars}
\NormalTok{        └── main.yml}
\end{Highlighting}
\end{Shaded}

\hypertarget{step-3---create-the-tasks-file}{%
\subsubsection{Step 3 - Create the Tasks
File}\label{step-3---create-the-tasks-file}}

The \texttt{main.yml} file in the tasks subdirectory of the role should
do the following:

\begin{itemize}
\tightlist
\item
  Make sure httpd is installed
\item
  Make sure httpd is started and enabled
\item
  Put HTML content into the Apache document root
\item
  Install the template provided to configure the vhost
\end{itemize}

\begin{quote}
\textbf{WARNING}

\textbf{The \texttt{main.yml} (and other files possibly included by
main.yml) can only contain tasks, \emph{not} complete playbooks!}
\end{quote}

Edit the \texttt{roles/apache\_vhost/tasks/main.yml} file:

\begin{Shaded}
\begin{Highlighting}[]
\PreprocessorTok{{-}{-}{-}}
\KeywordTok{{-}}\AttributeTok{ }\FunctionTok{name}\KeywordTok{:}\AttributeTok{ install httpd}
\AttributeTok{  }\FunctionTok{ansible.builtin.yum}\KeywordTok{:}
\AttributeTok{    }\FunctionTok{name}\KeywordTok{:}\AttributeTok{ httpd}
\AttributeTok{    }\FunctionTok{state}\KeywordTok{:}\AttributeTok{ latest}

\KeywordTok{{-}}\AttributeTok{ }\FunctionTok{name}\KeywordTok{:}\AttributeTok{ start and enable httpd service}
\AttributeTok{  }\FunctionTok{ansible.builtin.service}\KeywordTok{:}
\AttributeTok{    }\FunctionTok{name}\KeywordTok{:}\AttributeTok{ httpd}
\AttributeTok{    }\FunctionTok{state}\KeywordTok{:}\AttributeTok{ started}
\AttributeTok{    }\FunctionTok{enabled}\KeywordTok{:}\AttributeTok{ }\CharTok{true}
\end{Highlighting}
\end{Shaded}

Note that here just tasks were added. The details of a playbook are not
present.

The tasks added so far do:

\begin{itemize}
\tightlist
\item
  Install the httpd package using the yum module
\item
  Use the service module to enable and start httpd
\end{itemize}

Next we add two more tasks to ensure a vhost directory structure and
copy html content:

\begin{Shaded}
\begin{Highlighting}[]
\KeywordTok{{-}}\AttributeTok{ }\FunctionTok{name}\KeywordTok{:}\AttributeTok{ ensure vhost directory is present}
\AttributeTok{  }\FunctionTok{ansible.builtin.file}\KeywordTok{:}
\AttributeTok{    }\FunctionTok{path}\KeywordTok{:}\AttributeTok{ }\StringTok{"/var/www/vhosts/\{\{ ansible\_hostname \}\}"}
\AttributeTok{    }\FunctionTok{state}\KeywordTok{:}\AttributeTok{ directory}

\KeywordTok{{-}}\AttributeTok{ }\FunctionTok{name}\KeywordTok{:}\AttributeTok{ deliver html content}
\AttributeTok{  }\FunctionTok{ansible.builtin.copy}\KeywordTok{:}
\AttributeTok{    }\FunctionTok{src}\KeywordTok{:}\AttributeTok{ web.html}
\AttributeTok{    }\FunctionTok{dest}\KeywordTok{:}\AttributeTok{ }\StringTok{"/var/www/vhosts/\{\{ ansible\_hostname \}\}/index.html"}
\end{Highlighting}
\end{Shaded}

Note that the vhost directory is created/ensured using the \texttt{file}
module.

The last task we add uses the template module to create the vhost
configuration file from a j2-template:

\begin{Shaded}
\begin{Highlighting}[]
\KeywordTok{{-}}\AttributeTok{ }\FunctionTok{name}\KeywordTok{:}\AttributeTok{ template vhost file}
\AttributeTok{  }\FunctionTok{ansible.builtin.template}\KeywordTok{:}
\AttributeTok{    }\FunctionTok{src}\KeywordTok{:}\AttributeTok{ vhost.conf.j2}
\AttributeTok{    }\FunctionTok{dest}\KeywordTok{:}\AttributeTok{ /etc/httpd/conf.d/vhost.conf}
\AttributeTok{    }\FunctionTok{owner}\KeywordTok{:}\AttributeTok{ root}
\AttributeTok{    }\FunctionTok{group}\KeywordTok{:}\AttributeTok{ root}
\AttributeTok{    }\FunctionTok{mode}\KeywordTok{:}\AttributeTok{ }\DecValTok{0644}
\AttributeTok{  }\FunctionTok{notify}\KeywordTok{:}
\AttributeTok{    }\KeywordTok{{-}}\AttributeTok{ restart\_httpd}
\end{Highlighting}
\end{Shaded}

Note it is using a handler to restart httpd after an configuration
update.

The full \texttt{tasks/main.yml} file is:

\begin{Shaded}
\begin{Highlighting}[]
\PreprocessorTok{{-}{-}{-}}
\KeywordTok{{-}}\AttributeTok{ }\FunctionTok{name}\KeywordTok{:}\AttributeTok{ install httpd}
\AttributeTok{  }\FunctionTok{ansible.builtin.yum}\KeywordTok{:}
\AttributeTok{    }\FunctionTok{name}\KeywordTok{:}\AttributeTok{ httpd}
\AttributeTok{    }\FunctionTok{state}\KeywordTok{:}\AttributeTok{ latest}

\KeywordTok{{-}}\AttributeTok{ }\FunctionTok{name}\KeywordTok{:}\AttributeTok{ start and enable httpd service}
\AttributeTok{  }\FunctionTok{ansible.builtin.service}\KeywordTok{:}
\AttributeTok{    }\FunctionTok{name}\KeywordTok{:}\AttributeTok{ httpd}
\AttributeTok{    }\FunctionTok{state}\KeywordTok{:}\AttributeTok{ started}
\AttributeTok{    }\FunctionTok{enabled}\KeywordTok{:}\AttributeTok{ }\CharTok{true}

\KeywordTok{{-}}\AttributeTok{ }\FunctionTok{name}\KeywordTok{:}\AttributeTok{ ensure vhost directory is present}
\AttributeTok{  }\FunctionTok{ansible.builtin.file}\KeywordTok{:}
\AttributeTok{    }\FunctionTok{path}\KeywordTok{:}\AttributeTok{ }\StringTok{"/var/www/vhosts/\{\{ ansible\_hostname \}\}"}
\AttributeTok{    }\FunctionTok{state}\KeywordTok{:}\AttributeTok{ directory}

\KeywordTok{{-}}\AttributeTok{ }\FunctionTok{name}\KeywordTok{:}\AttributeTok{ deliver html content}
\AttributeTok{  }\FunctionTok{ansible.builtin.copy}\KeywordTok{:}
\AttributeTok{    }\FunctionTok{src}\KeywordTok{:}\AttributeTok{ web.html}
\AttributeTok{    }\FunctionTok{dest}\KeywordTok{:}\AttributeTok{ }\StringTok{"/var/www/vhosts/\{\{ ansible\_hostname \}\}/index.html"}

\KeywordTok{{-}}\AttributeTok{ }\FunctionTok{name}\KeywordTok{:}\AttributeTok{ template vhost file}
\AttributeTok{  }\FunctionTok{ansible.builtin.template}\KeywordTok{:}
\AttributeTok{    }\FunctionTok{src}\KeywordTok{:}\AttributeTok{ vhost.conf.j2}
\AttributeTok{    }\FunctionTok{dest}\KeywordTok{:}\AttributeTok{ /etc/httpd/conf.d/vhost.conf}
\AttributeTok{    }\FunctionTok{owner}\KeywordTok{:}\AttributeTok{ root}
\AttributeTok{    }\FunctionTok{group}\KeywordTok{:}\AttributeTok{ root}
\AttributeTok{    }\FunctionTok{mode}\KeywordTok{:}\AttributeTok{ }\DecValTok{0644}
\AttributeTok{  }\FunctionTok{notify}\KeywordTok{:}
\AttributeTok{    }\KeywordTok{{-}}\AttributeTok{ restart\_httpd}
\end{Highlighting}
\end{Shaded}

\hypertarget{step-4---create-the-handler}{%
\subsubsection{Step 4 - Create the
handler}\label{step-4---create-the-handler}}

Create the handler in the file
\texttt{roles/apache\_vhost/handlers/main.yml} to restart httpd when
notified by the template task:

\begin{Shaded}
\begin{Highlighting}[]
\PreprocessorTok{{-}{-}{-}}
\CommentTok{\# handlers file for roles/apache\_vhost}
\KeywordTok{{-}}\AttributeTok{ }\FunctionTok{name}\KeywordTok{:}\AttributeTok{ restart\_httpd}
\AttributeTok{  }\FunctionTok{ansible.builtin.service}\KeywordTok{:}
\AttributeTok{    }\FunctionTok{name}\KeywordTok{:}\AttributeTok{ httpd}
\AttributeTok{    }\FunctionTok{state}\KeywordTok{:}\AttributeTok{ restarted}
\end{Highlighting}
\end{Shaded}

\hypertarget{step-5---create-the-web.html-and-vhost-configuration-file-template}{%
\subsubsection{Step 5 - Create the web.html and vhost configuration file
template}\label{step-5---create-the-web.html-and-vhost-configuration-file-template}}

Create the HTML content that will be served by the webserver.

\begin{itemize}
\tightlist
\item
  Create an web.html file in the ``src'' directory of the role,
  \texttt{files}:
\end{itemize}

\begin{Shaded}
\begin{Highlighting}[]
\CommentTok{\#\textgreater{} echo \textquotesingle{}simple vhost index\textquotesingle{} \textgreater{} \textasciitilde{}/ansible{-}files/roles/apache\_vhost/files/web.html}
\end{Highlighting}
\end{Shaded}

\begin{itemize}
\tightlist
\item
  Create the \texttt{vhost.conf.j2} template file in the role's
  \texttt{templates} subdirectory.
\end{itemize}

The contents of the \texttt{vhost.conf.j2} template file are found
below.

\begin{Shaded}
\begin{Highlighting}[]
\NormalTok{\# \{\{ ansible\_managed \}\}}

\NormalTok{\textless{}VirtualHost *:8080\textgreater{}}
\NormalTok{    ServerAdmin webmaster@\{\{ ansible\_fqdn \}\}}
\NormalTok{    ServerName \{\{ ansible\_fqdn \}\}}
\NormalTok{    ErrorLog logs/\{\{ ansible\_hostname \}\}{-}error.log}
\NormalTok{    CustomLog logs/\{\{ ansible\_hostname \}\}{-}common.log common}
\NormalTok{    DocumentRoot /var/www/vhosts/\{\{ ansible\_hostname \}\}/}

\NormalTok{    \textless{}Directory /var/www/vhosts/\{\{ ansible\_hostname \}\}/\textgreater{}}
\NormalTok{  Options +Indexes +FollowSymlinks +Includes}
\NormalTok{  Order allow,deny}
\NormalTok{  Allow from all}
\NormalTok{    \textless{}/Directory\textgreater{}}
\NormalTok{\textless{}/VirtualHost\textgreater{}}
\end{Highlighting}
\end{Shaded}

\hypertarget{step-6---test-the-role}{%
\subsubsection{Step 6 - Test the role}\label{step-6---test-the-role}}

You are ready to test the role against \texttt{node}. But since a role
cannot be assigned to a node directly, first create a playbook which
connects the role and the host. Create the file
\texttt{test\_apache\_role.yml} in the directory
\texttt{\textasciitilde{}/ansible-files}:

\begin{Shaded}
\begin{Highlighting}[]
\PreprocessorTok{{-}{-}{-}}
\KeywordTok{{-}}\AttributeTok{ }\FunctionTok{name}\KeywordTok{:}\AttributeTok{ use apache\_vhost role playbook}
\AttributeTok{  }\FunctionTok{hosts}\KeywordTok{:}\AttributeTok{ node}
\AttributeTok{  }\FunctionTok{become}\KeywordTok{:}\AttributeTok{ }\CharTok{True}
\AttributeTok{  }\FunctionTok{pre\_tasks}\KeywordTok{:}
\AttributeTok{    }\KeywordTok{{-}}\AttributeTok{ }\FunctionTok{ansible.builtin.debug}\KeywordTok{:}
\AttributeTok{        }\FunctionTok{msg}\KeywordTok{:}\AttributeTok{ }\StringTok{\textquotesingle{}Beginning web server configuration.\textquotesingle{}}
\AttributeTok{  }\FunctionTok{roles}\KeywordTok{:}
\AttributeTok{    }\KeywordTok{{-}}\AttributeTok{ apache\_vhost}
\AttributeTok{  }\FunctionTok{post\_tasks}\KeywordTok{:}
\AttributeTok{    }\KeywordTok{{-}}\AttributeTok{ }\FunctionTok{ansible.builtin.debug}\KeywordTok{:}
\AttributeTok{        }\FunctionTok{msg}\KeywordTok{:}\AttributeTok{ }\StringTok{\textquotesingle{}Web server has been configured.\textquotesingle{}}
\end{Highlighting}
\end{Shaded}

Note the \texttt{pre\_tasks} and \texttt{post\_tasks} keywords.
Normally, the tasks of roles execute before the tasks of a playbook. To
control order of execution \texttt{pre\_tasks} are performed before any
roles are applied. The \texttt{post\_tasks} are performed after all the
roles have completed. Here we just use them to better highlight when the
actual role is executed.

Now you are ready to run your playbook:

\begin{Shaded}
\begin{Highlighting}[]
\ExtensionTok{[student@controller}\NormalTok{ ansible{-}files]$ ansible{-}navigator run test\_apache\_role.yml}
\end{Highlighting}
\end{Shaded}

Run a curl command against \texttt{node} to confirm that the role
worked:

\begin{Shaded}
\begin{Highlighting}[]
\ExtensionTok{[student@controller}\NormalTok{ ansible{-}files]$ curl }\AttributeTok{{-}s}\NormalTok{ http://10.3.48.[100+PARTICIPANT\_ID]:8080}
\ExtensionTok{simple}\NormalTok{ vhost index}
\end{Highlighting}
\end{Shaded}

Congratulations! You have successfully completed this exercise!

\hypertarget{troubleshooting-problems}{%
\subsection{Troubleshooting problems}\label{troubleshooting-problems}}

Did the final curl work? You can see what ports the web server is
running by using the ss command:

\begin{Shaded}
\begin{Highlighting}[]
\CommentTok{\#\textgreater{} sudo ss {-}tulpn | grep httpd}
\end{Highlighting}
\end{Shaded}

There should be a line like this:

\begin{Shaded}
\begin{Highlighting}[]
\ExtensionTok{tcp}\NormalTok{   LISTEN 0      511                }\PreprocessorTok{*}\NormalTok{:8080               }\PreprocessorTok{*}\NormalTok{:}\PreprocessorTok{*}\NormalTok{    users:}\ErrorTok{(}\KeywordTok{(}\StringTok{"httpd"}\ExtensionTok{,pid=182567,fd=4}\KeywordTok{)}\ExtensionTok{,}\ErrorTok{(}\StringTok{"httpd"}\ExtensionTok{,pid=182566,fd=4}\KeywordTok{)}\ExtensionTok{,}\ErrorTok{(}\StringTok{"httpd"}\ExtensionTok{,pid=182565,fd=4}\KeywordTok{)}\ExtensionTok{,}\ErrorTok{(}\StringTok{"httpd"}\ExtensionTok{,pid=182552,fd=4}\KeywordTok{))}
\end{Highlighting}
\end{Shaded}

Pay close attention to the fifth column of the above output. It should
be \texttt{*:8080}. If it is \texttt{*:80} instead or if it is not
working, then make sure that the \texttt{/etc/httpd/conf/httpd.conf}
file has \texttt{Listen\ 8080} in it. This should have been changed by
\href{../1.5-handlers}{Exercise 1.5}
